\section{Supplement: Chemical heterogeneity and the approach to equilibrium during the experiments}
\label{ApproachEq_Sup}
\subsection{Attainment of equilibrium in the Canary Islands rocks}
The Canary Islands trace-element partition-coefficients presented here were determined from euhedral, blade-shaped crystals free of melt inclusions and chemical zonation. The corresponding quenched melt was in direct contact with these crystals and shows no zonation in backscattered electron images (Fig. \ref{3_ChemTransect}). While equilibrium conditions are challenging to confirm for a natural volcanic system, the euhedral forms, chemical homogeneity of crystals, and congruency between samples from separate eruptions suggest that the crystals grew in a stable environment, and were not subject to chemical or physical perturbations during growth (Fig. \ref{6_D_Spider}).	

\subsection{Attainment of equilibrium in the experiments}
%See Adam1993 start of discussion. There is a long section entitled "Heterogeneity and approach to equilibrium during experiments"
Experiments used to determine trace-element partition-coefficients must have attained, or at least closely approached, chemical equilibrium. Unfortunately no experiments are able to determine equilibrium trace-element partition-coefficients \emph{sensu stricto} because reversal experiments, where a clinopyroxene re-equilibrates with a melt, are not possible owing to sluggish diffusion of most elements through the clinopyroxene structure \citep{VanOrman2001,Zhang2010}.
     The following two sections discuss some analytical and experimental biases that must be considered when determining mineral-melt trace-element partition-coefficients from crystallisation experiments.

%%%%%%%%%
%SEE REMOVED MATERIAL TEX
%%%%%%%%%
\subsubsection{The formation of diffusive boundary layers}
A potential barrier to chemical equilibration during crystallisation is the formation of a diffusive boundary layer within the melt adjacent to growing crystals. In a perfect equilibrium case there are no compositional gradients in the melt at any time during crystal growth. However in reality the growth of crystals depletes compatible elements from the melt and residually enriches incompatible elements. 
    Theoretically, this process of crystal growth results in formation of a (potentially ephemeral) diffusive boundary layer in the melt directly adjacent to the crystal that is depleted with respect to compatible elements and enriched with respect to those that are incompatible \citep{Lu1995}. The composition of such a diffusive boundary layer depends on both the relative enrichment or depletion of elements during the crystallisation process, and the rate at which these elements diffuse through the melt. Rapidly-diffusing elements with partition-coefficients close to unity will have concentrations closest to that of the bulk melt.

Experiments designed to investigate trace-element-partitioning behaviour might employ slow cooling rates to limit the development of diffusive boundary layers, thus forming crystals from melt that is closer in composition to that of the bulk experiment. Such experiments then run into another problem, in that significant crystallisation may occur at temperatures above that of the final run temperature. Rapidly-cooled experiments circumnavigate this issue, but may form relatively more pronounced diffusive boundary layers during crystal growth that become `flattened out' during the homogenisation stage of the experiment.

Numerous diffusion data have been gathered for silicate melts over the past few decades, and a comprehensive review is given by \citet{Zhang2010}. Diffusion of trace-elements through water-saturated peralkaline melts is rapid, owing to their depolymerised structure. For example Lanthanum diffusion-coefficients are 6 orders of magnitude higher than for water-saturated granitic compositions of a similar temperature \citep[compare][]{Rapp1986,Behrens2009}. This rapid diffusion serves to minimise the formation of diffusive boundary layers adjacent to growing crystals in our experiments.
Coupled diffusion mechanisms complicate the application of measured single-element diffusion-coefficients to a crystallising system \citep{Grove1984,Liang1994,Costa2003}. Here, the diffusive flux of trace-elements may be coupled to gradients in major-element concentration within the melt. 

%%%%%%%%%%%%%%%

To investigate the impact of diffusive effects on trace-element-partitioning between clinopyroxene and melt, \citet{Mollo2013} performed crystallisation experiments on trachybasaltic melts at a range of cooling rates (2.5--50\dgC/ hr). Rapid cooling rates result in depletions of Si, Ca and Mg in the clinopyroxene that are compensated for by enrichments in Al, Na and Ti. Regardless of cooling rate, Ounma parabolae could be fitted through isovalent sets of partition-coefficients, indicating that crystal-lattice-effects dominated over those associated with the formation of diffusive boundary layers and that local equilibrium was achieved at the time of crystallisation.
    % In their rapidly-cooled experiments, \citet{Mollo2013} found apparent clinopyroxene/melt trace-element-partition coefficients that varied systematically with the \ce{^{IV}Al} content of the clinopyroxene, regardless of cooling rate.
    In their rapidly-cooled experiments \citet{Mollo2013} found apparent clinopyroxene/melt trace-element partition-coefficients that varied with identical crystal-chemical systematics to true equilibrium partition coefficients, the magnitude of both sets of trace-element partition-coefficients following the \ce{^{IV}Al} content of the clinopyroxene (ibid., their Fig. 9). 
Deviations of the partition coefficient of several orders of magnitude can be obtained only when rapidly growing crystals entrap small portions of the diffusive boundary layer that are found as minute melt inclusions randomly distributed in the mineral phase \citep{Kennedy1993}. In this extreme case, partitioning behaviour is obviously influenced by contamination phenomena and no Onuma parabolae can be derived. As Onuma parabolae could successfully be fitted through partitioning data for all of our presented experiments (see following sections), and no melt inclusions were observed in optical and electron imaging, we infer that our data were not affected by the presence of such melt inclusions, and that they may be compared directly with partitioning data derived from experiments that employed slower cooling rates.

 %S. Mollo, email via Vincent 29Jan2018:
 %Also, the following point of the Reviewer2 is not entirely correct: "The presence of major-element concentration gradients around the natural crystals investigated leads me to expect even larger REE concentration gradients around the growing crystals. If this is true, the reported partition coefficients are minima and maxima rather than representative of equilibrium. In this case, the data are inappropriate for lattice-strain modeling, particularly as a part of a larger dataset comprising experiments at equilibrium". Reading through Mollo et al. (2013) published in CTMP, you will see that the lattice-strain modeling can be successfully applied under rapid growth conditions when a diffusive boundary layer develops at the crystal-melt interface. This implies that the "true" partition coefficient changes as an "apparent" partition coefficient. If the lattice-strain equation fits the Onuma diagram (i.e., partition coefficient versus ionic radius), it can be concluded that local equilibrium condition was attained at the time of crystallization. Moreover, apparent partition coefficients change by quantities of the same order of magnitude of the true partition coefficients due to crystallographic effects (Mollo et al., 2013). Deviations of the partition coefficient of several orders of magnitude can be obtained only when rapidly growing crystals may entrap small portions of the diffusive boundary layer that are found as minute melt inclusions randomly distributed in the mineral phase (Kennedy et al., 1993). But in this case the partition coefficient is obviously influenced by contamination phenomena and no Onuma parabolas can be derived.
 
 %% VvH email 29Jan2018: Also, a quick check of the Ce data from the probe shows that the max deviation between rim composition and grain average would be a factor of 1.5 (most are between 1.1 and 1.3). So this does not seem to be an issue at all. Lang unfortunately did not send me all the data, but I?ll gather the remainder on Tuesday. I doubt that the other data will change anything though. 
 
%%%%%%%%%


\subsubsection{Chemical zonation in the experiment clinopyroxene: Theoretical framework}
Trace-elements diffuse slowly through the clinopyroxene structure relative to that of the melt \citep{VanOrman2001, Zhang2010}, therefore no re-equilibration of trace-elements takes place on an experimental time scale. Strictly speaking, clinopyroxene only record true equilibrium conditions at their outermost rim. 
     Experiments designed for the derivation of equilibrium partition coefficients ideally minimise bias by limiting the fraction of crystallisation, producing minerals that are as homogeneous as possible. Currently available \textit{in situ} analytical techniques for trace-element abundances, such as LA-ICP-MS and SIMS, are limited in terms of minimum beam-size to $\sim$10 $\mu$m; chemical zonation, however subtle, will be continuous from the core to the rim of the mineral. Consequently, no experimentally-derived partition-coefficients record chemical equilibrium \textit{sensu stricto}, but properly conducted experiments may closely approximate this state.

Because only the very rim of a crystal records chemical equilibrium with the adjacent melt, and some internal portions of the minerals must be sampled during \textit{in-situ} analyses all experimentally-determined trace-element partition-coefficients are biased toward higher values for compatible elements and lower values for incompatible elements. The magnitude of these biases depends on the fraction of crystallisation in the experiment, the true equilibrium partition-coefficient of that element, as well as the proportion of each growth-zone sampled during the \textit{in-situ} analysis. Fortunately \textit{in-situ} chemical analyses preferentially sample the mantle and rim of zoned crystals because few analyses section a crystal perfectly through the core. As a result there is a sampling bias toward equilibrium mineral compositions.

	% % Fedele: You never talked about average values... [CB: comment addressed, removed.]

	Consider a hypothetical experimental system in which 20\% of the melt crystallises as a single mineral, and where the chemical analyses of that mineral are truly bulk averages of that mineral composition. An incompatible element with a true equilibrium partition coefficient of 0.1 would return a measured partition coefficient of 0.09, a small bias because the concentration of that incompatible element in the melt changed only subtly during the course of crystallisation. For compatible elements with true $D_i$ values of $\sim 10$, measured partition coefficients can be a factor of 2--3 higher than true partition coefficients, because their concentration in the melt changes more than an incompatible element during the course of crystallisation.

    Further complexity is introduced in systems that crystallise multiple minerals simultaneously. In the case of experiment M3\ce{_2}, the REE are compatible in clinopyroxene, but are incompatible in biotite and oxides \citep[e.g.][]{Mahood1990,Schmidt1999}. The REE have therefore been residually enriched in the melt phase by the crystallisation of biotite and oxide minerals, while simultaneously being depleted from the melt by crystallisation of clinopyroxene. These two competing processes serve to minimise the effect of fractional crystallisation on the concentration of trace-elements in the melt and consequently derived REE partition-coefficients between clinopyroxene and melt will be closer to true equilibrium values.

\subsubsection{Implications of the cerium zonation across the experiment clinopyroxene}
    Electron-microprobe analyses offer a smaller minimum beam-size than LA-ICP-MS systems at the expense of precision and of number of elements that may be analysed simultaneously. This higher spatial resolution permitted investigation of the zonation of Ce concentrations within the experiment clinopyroxene, with Ce as a proxy for the other compatible elements.

% As presented above, the effect of sector-zoning on the Ce concentration in the experiment clinopyroxene strongly outweighs that of concentric growth zoning. 

The magnitude of concentric growth zoning in the experiment clinopyroxene was examined by averaging bulk and rim compositions across multiple sector-zoned grains within each experiment. The median Ce counts divided by the rim Ce counts was 1.04 for experiment M3\ce{_2}, 1.08 for experiment M5 and 1.12 for the fluorine-bearing experiment M3-1.25F \citep[see][]{Beard_PhD_thesis}. Contrary to theory discussed above, the subtly (3\%) crystallised experiment M5 shows a greater variation in core-to-rim Ce content than the more heavily (?? TO ADD\%) crystallised experiment M3\ce{_2}. Furthermore, the $D_{Ce}^{cpx/melt}$ for experiment M3\ce{_2} (6.2) is approximately double that for experiment M5 (see discussion below), which should further promote the formation of Cerium zonation in the clinopyroxene during crystallisation. Consequently it is possible that many existing experimentally-derived trace-element partition-coefficients are systematically offset from true equilibrium values.

If Cerium is used as a proxy for the behaviour of compatible elements in our experimental system, then apparent partition coefficients derived for these elements via LA-ICP-MS are systematically offset to higher values by 4--8\%. Such a systematic bias is small, relative to the variation in clinopyroxene-melt partition coefficients within our sample set, as well as in the literature. The effect of crystal zonation on incompatible element concentrations was not measured, but as outlined in the theoretical framework above, this effect should be smaller than that for the compatible elements. No correction factor has been applied to the data presented in the figures below, or in Table \ref{D_table} and \ref{Ae_SupTbl}. 