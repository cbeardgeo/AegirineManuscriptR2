
Several lines of evidence indicate that our experiments preserve conditions close to chemical equilibrium. 
	The experimental charges underwent small degrees of crystallisation ($\le$ 22\%), resulting in small changes to liquid compositions during the run (Fig. \ref{TAS}). 


Clinopyroxene produced in our experiments do indeed display chemical zonation  
% Fedele: As said in a previous comment, this does not seem to be completely true or, at least, it is not completely clear based on what you report. But, most importantly, I would expect that some detailed data presentation is reported before coming to such conclusion (see  previous comments). 
	%, suggesting that the analysed trace-element compositions of bulk crystals are similar to that of the rims and are thus representative of equilibrium element-partitioning at the respective conditions 
	(Figs. \ref{2_Elementmap}, \ref{Px_Maj}).
	%Where chemical zonation is present within the clinopyroxene crystals, the concentration of compatible elements is higher in the core than the rim, whereas for incompatible elements, the reverse case is true. %	A correction term to equilibrium for rim compositions.
	

	Such a small range would not be expected if significant zoning were present, because the sectioning level of clinopyroxene varies on the polished surface and with it the relative contributions of core and rim in the analyzed volume. 
	
	

	Because our results are consistent regardless of the cooling path that was used for the experiments (see Table \ref{Ae_SupTbl}), and further that quenched melts are free of chemical zonation (Fig. \ref{2_Elementmap}), we suggest that the melt composition remained homogeneous during crystallisation, and that kinetic effects associated with diffusion of elements toward or away from the crystal-melt interface \citep[e.g.][]{Lofgren2006} were smaller than the analytical precision of our setup.
	
	%The rate at which crystals grow is also recorded in their composition; Crystallisation depletes the surrounding melt with respect to compatible elements while residually enriching those that are incompatible. The composition of the melt from which a crystal is growing is therefore dependent on the rate at which elements may diffuse toward or away from the crystal-melt boundary \citep[e.g.][]{Lofgren2006}. Rapid growth will offset effective partition coefficients toward unity, especially for slow-diffusing elements such as the HFSE \citep[cf.][]{Zhang2010}.
	
Most experiments display a uniform distribution of crystals, suggesting that melts were homogeneous at the beginning of crystallisation, that experimental charges ran under isothermal conditions, and that gravitational separation of crystals was minimal. As charges were fused at superliquidus conditions for a minimum of 16 hrs, the \ce{H2O} content of the melts should have been homogeneous and \textit{f}\ce{H2} (thus \fO) equilibrated with the gas pressure medium on initiation of crystallisation \citep{Gaillard2002}.

Near-duplicate experiments M4$_3$ and M4$_4$ have similar phase relations
% Fedele: What do you mean? Where can this be observed? Please elucidate.
%CB add another supplementary figure?
, crystal compositions and glass compositions, despite running for different times (45 hr vs. 49 hr). Additionally, two experiments that varied only in fluorine content (M3\ce{_{0.3F}}, with 0.3 wt.\% fluorine, and M3$_2$, fluorine-free) also generated similar mineral compositions and have near-identical element-partitioning behaviour \citep{Beard_PhD_Thesis}.% \citep[See][]{Beard_inprep_2}. ADD WHEN SUBMITTED

Most experiments produced euhedral crystals. Only those on NLS compositions show hopper and swallowtail textures, indicative of diffusion-limited crystal growth \citep{Walker1976, Lofgren1989, Shea2013}. Such textures record growth that was limited with respect to major-elements, and do not necessarily record equilibrium element-partition coefficients for trace-elements. Diffusion-limited growth theoretically converges apparent clinopyroxene partition coefficients toward unity \citep[e.g.][]{Blundy1998}, suggesting that in these experiments, the partition coefficients in excess of one should be treated as minimum values, and those below as maxima. However the trends in partitioning behaviour are in excellent agreement with our experiments that grew euhedral crystals, suggesting minimal disequilibria effects (see discussion of growth rates below and \citealt{Mollo2013}).



%%%%%%%%%%%%%%%%%%%%%%%
%%%%%%%%%%%%%%%%%%%%%%%
CRYSTAL ZONATION













Important because growth of crystals depletes the melt with respect to compatible elements, and slow in-crystal diffusion of trace-elements results in concentric growth-zonation of trace-element concentrations within crystals with compatible elements enriched in the cores and incompatible elements enriched in the rims. The rim compositions record true chemical equilibrium with the adjacent melt, but simultaneous measurement of all trace-elements of interest in the rims of crystals commonly reaches the limits of modern analytical techniques.








	
%	VvH v13.1: We need to add a statement here on the preservation of the systematics in zoning. Something along the lines of: ?Finally, assuming that equilibrium partitioning takes place throughout growth, zoning predominantly affects absolute $D_i$ values, whereas the trends with P, T and mineral composition are preserved. Zoning therefore does not impact our conclusions on the crystal-chemical and P-T controls on partitioning, nor the models that we derive for this. It would merely move absolute values by up to 25\% for compatible elements.?
	
	

%\textbf{CB v13: Should the LA data be corrected for this fractionation effect? This would require that transects were measured for all experiments to get core-rim profiles for each, so probably unrealistic. Additionally, more compatible elements should be affected more strongly by this fractional crystallisation effect, relative to elements with D values close to unity. Perhaps this is why the low-aegirine clinopyroxene return such high E M2 values?}

%JS v.13: What do you think the actual bias is for REE and other elements? 25% ? Lower, and if so, how much lower? 25% is not negligible. The question is how practical doing these transects would be.
%CB v13: Ultimately, measuring EPMA transects isn't practical for all experiments and wouldn't even give us a solid set of numbers. This is because the concentration of each element in the residual melt will evolve based on its true partition coefficient, with more compatible elements showing stronger bias during fractionation than incompatible elements (see discussion, paragraph beginning `we modelled...'). 25% is a worst case scenario, and is ultimately quite small compared to the reproducibility of our empirical model in Fig. 13 (+250%/290% @ 95% confidence interval). Similar published models are about +/- 200%, but are calibrated for use over much, much narrower ranges of composition.

------------------------



In most experiments, the small clinopyroxene crystals ablated quickly and a mixture of clinopyroxene and quenched melt were analysed by the ICP-MS. Unmixing of the resultant signal necessarily requires averaging of several spot or line analyses to return compositional data with residuals sufficiently small as to be useful (discussed above, and in Supplementary Figure \ref{LaserMix}). However, unmixing of subsets point or line analyses return similar compositions, suggesting



However, the melt reservoir from which crystals were growing is finite in mass and slow in-crystal diffusion has preserved major-element zonation in the pyroxenes, both in concentric growth zonation and in sector zonation patterns.

   The small magnitude of this growth zonation effect, relative to the worst case model scenario discussed above suggests that, in the case of incompatible elements, the partition coefficients derived from our experiments will be within analytical uncertainty of the true equilibrium values, whereas for compatible elements, they would be well within an order of magnitude of true values.
    

    LA-ICP-MS partition coefficients that we present are therefore less strongly biased toward high values than the 8\% value suggested by the EPMA data for Ce discussed above.
    Finally, assuming that equilibrium partitioning takes place throughout growth, zoning predominantly affects absolute $D_i$ values, whereas the relationship between $D_i$ values and P, T and mineral composition are preserved. Zoning therefore does not impact our conclusions on the crystal-chemical and P-T controls on partitioning, nor the models that we derive for this. It would merely move absolute values by up to 12\% for compatible elements.













However, some variation in the trace-element content of the experiment clinopyroxene does exist between their cores and rims. For example in the case of Ce, a compatible element in most of our experiments, the concentration is higher at the cores of experiment clinopyroxene than at the rim. 




Due to growth of crystals from a limited experimental volume. 

In a `perfect' experiment, the 


% % Fedele: In any case, your trace element data looks definitely more homogeneous (though I think there still is some variability for some elements), which could be considered somewhat strange, given that major element data are not. However, at a more detailed inspection of your Table SM1, it seems that the cpx analysed for trace elements are a selection of the entire dataset, actually featuring limited chemical variability also for major elements.
%CB: is that so? That wasn't deliberate
%Fedele: However, I'm not sure this has been ever stated in the manuscript (neither how this selection has been done), giving the impression that analysed minerals are not homogeneous, as evident from figure 4. 

% % Fedele: In conclusion, I think some additional discussion and details about all these aspects should be provided, possibly accompanied by additional figures that could be of great help in order to make it easier for the reader to visualise all such things (rather than downloading the ESM material and carefully checking the tables). Maybe some chondrite-normalized multielement plots can be of use.

Where the associated laser-ablation output counts were stable, indicating analyses of pure clinopyroxene, as opposed to a mixture of clinopyroxene and quenched melt, the


time-integrated averages were possible



Clinopyroxene has the general formula M2M1\ce{T2O6}, with the M1 and M2 sites incorporating most of the trace-elements of geological interest \citep{Morimoto1989}. 



%%%%%%%%
MORE EQUILIBRATION STUFF


    The concentration of trace-elements in the melt may be residually enriched by the crystallisation of one phase while simultaneously being depleted by another. An example is 	

	% % Fedele: Not clear how you modelled this. Can you give some more details?
	
	Variability in the concentration of Ti in clinopyroxene from experiment M3\ce{_2}, one of our most completely crystallised charges, suggests that this model strongly exaggerates the effect of chemical zoning on partition coefficients, relative to our experiments. If maximum and minimum concentrations of Ti in the clinopyroxene are used to calculate partition coefficients, the range of derived $D_{Ti}$ values is 12--16.	
 In our experiments, the trace-element composition of the melt does not vary as much as in this worst-case scenario model, partially because additional mineral phases have crystallised alongside clinopyroxene.

 This modelling considered, the experiment partition coefficients that we report for incompatible elements should be reliable within analytical uncertainty, whereas the $D$ values for compatible elements should be considered as maxima. Based on the systematics of Ti partitioning in experiment M$3_2$, compatible trace-element partition coefficients are probably overestimated by no more than 30\%.	
